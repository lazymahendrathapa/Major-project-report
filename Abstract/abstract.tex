\newpage
\section*{ABSTRACT}
\addcontentsline{toc}{section}{\numberline{} ABSTRACT}

This report describes and documents all the aspects and working functionality of our final year project titled “Music Classification System
Based on Genre and Mood”. The project is part for the curriculum for the subject Major Project under the course of final year of B.E. in Computer Engineering.
As the title itself describes the overall aim of the project is to develop a system capable of classifying music based on Genre and mood, with the availability of large
number of digital media and the disorder introduced being the primary motivation.\\  
\\
The methodology used is that of a modular system consisting of two main stages. The first stage involves the preprocessing of the raw audio
data resulting in the extraction of a number of features pertaining to music signal: Intensity, MFCC, rhythm, pitch. Each feature extractor
reduces the information content in the raw data to a vector in a small number of dimensions. Or in other words we can say that feature extractor 
analyses the music signal and extracts its respective features compatible for further processing. It requires intensive knowledge of digital signal
analysis and processing, signal sampling,etc. The second stage comprises of all the machine learning portion. In it, the set of feature vectors are classified(indexed) into certain clusters
by the use of certain algorithms: K-means, Support Vector Machines and  Artificial Neural Networks. This technically requires knowledge of all those respective algorithms.\\ 
\\
This report also documents our approach towards the system development following the various aspects of Software Engineering. UML diagrams have been
used to model the entire system and ERD diagrams have been used to show the relationship between the various entities in our system and iterative 
development method was chosen for the development of our system. Java language along with spring framework was used to build our whole system along
with the GUI.\\


