\newpage
\section*{ABSTRACT}
\addcontentsline{toc}{section}{\numberline{} ABSTRACT}

With the rise of the internet, an overwhelmingly large number of digital media has become available to us in this day and age.
So, systems and tools to organize or sort through such large audio collections have become a necessity. 
These tools can also be important for Music Information Retrieval, queries based on audio, music search systems and even to the understanding of music itself.
\\
This project presents one such system capable of classifying music based on Genre and mood.
\\
For the Genre-based classification, we used five genres: Classical, Hiphop, Jazz, Pop. Rock, thus making it a five-way classification problem.
For the Mood-based classification, we used a mood model that maps mood onto a two dimensional space along the axes of \textit{arousal} and \textit{valence}.
So, this classification consisted of two binary classification, one for each of the axes.
\\
The system consists of two main stages: feature extraction and classification. 
In the first stage, features that provide a concise representation of information contained in song files, are extracted. 
We have extracted and studied ten such features in this project.
In the second stage, machine learning algorithms--Support Vector Machine and Artificial Neural Network--are used to classify the songs using those features.
\\
After analyzing the individual and combined effectiveness of the various features, as well as various parameters for the learning algorithms, we managed to achieve
the maximum overall accuracy of 88 per cent for the Genre Classification problem.
Similarly, for the mood based classification problem, we achieved around 73 per cent (Arousal) and 67 per cent (Valence) overall accuracy for the two axes.

